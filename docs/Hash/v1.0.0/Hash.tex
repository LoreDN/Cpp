\documentclass[a4paper, 10pt]{article}

% import packages
\usepackage[utf8]{inputenc}
\usepackage[T1]{fontenc}
\usepackage[english]{babel}
\usepackage{xcolor}
\usepackage[hidelinks]{hyperref}
\usepackage{tocloft}
\usepackage{varwidth}
\usepackage{listings}

% margin settings
\setlength{\oddsidemargin}{0pt}
\setlength{\evensidemargin}{0pt}
\setlength{\topmargin}{0pt}
\setlength{\headheight}{0pt}
\setlength{\textwidth}{16cm}
\setlength{\textheight}{24cm}

% table of contents settings
\setlength{\cftbeforesecskip}{0.5cm}
\setlength{\cftbeforesubsecskip}{0.25cm}

% command \link definition
\newcommand{\link}[1]{\textcolor{blue!80}{\underline{\href{#1}{#1}}}}

% command \code definition
\newcommand{\code}[1]{\colorbox{gray!20}{\texttt{#1}}}

% command \libcategory definition
\newcommand{\libcategory}[1]{

    \begin{minipage}{\textwidth}
        
        \noindent \par \centering
        \phantomsection \addcontentsline{toc}{subsection}{#1} 
        {\color{green!80} \rule{\textwidth}{1pt}} \par \vspace{0.6em}
        {\large \color{green!80} #1} \\[0.2em]
        {\color{green!80} \rule{\textwidth}{1pt}} \par \vspace{0.4em}

    \end{minipage}

}

% command \libfunction definition
\newcommand{\libfunction}[1]{

    \par \noindent
    \begin{minipage}{\textwidth}
        
        \noindent \centering \par \vspace{0.75cm}
        {\small \colorbox{black}{\textcolor{white}{\texttt{#1}}}}

    \end{minipage} \vspace{0.15cm}

}

% command \libclass definition
\newcommand{\libclass}[1]{

    \vspace{0.75cm} \par \noindent
    \makebox[\textwidth][c]{

        \setlength{\fboxrule}{0pt}
        \setlength{\fboxsep}{8pt}
        \fcolorbox{black}{black}{

            \begin{varwidth}{\textwidth}
                
                \textcolor{white}{\texttt{#1}}
                
            \end{varwidth}

        }

    } \par \vspace{0.5cm}

}

% command \midtilde definition
\DeclareRobustCommand{\midtilde}{\raisebox{-0.75ex}{\textasciitilde}}

% pagestyle setting
\makeatletter

    \def \ps@plain{

        \let \@oddhead \@empty \@evenhead \@empty
        \def \@oddfoot{

            {\color{black!70} \makebox[0pt][l]{Code Documentation}} \hfil
            {\color{black!70} \makebox[0pt][c]{C++ --- Hash}} \hfil
            {\color{black!70} \thepage}
        
        } \def \@evenfoot{\@oddfoot}

    }

\makeatother \pagestyle{plain}

% Start of the documentation
\begin{document}

    % First Page
    \begin{minipage}{\textwidth}
        
        \centering 
        \underline{Github:} LoreDN \\[0.2em]
        \underline{Author:} Lorenzo Di Napoli \\[0.2em]
        \underline{Repository:} \href{https://github.com/LoreDN/Cpp}{https://github.com/LoreDN/Cpp} \\[0.5em] 

        \rule{\textwidth}{1pt} \par \vspace{0.7em}
        {\Huge Hash\_v1.0.0 --- Documentation} \\[0.5em]
        \rule{\textwidth}{1pt} \par \vspace{0.5em}
        
    \end{minipage}

    \tableofcontents

    % Library Description
    \pagebreak
    \section*{Library Description} \addcontentsline{toc}{section}{Library Description}

        This library implements some usefull Hash-Table based Data-Structures, such as Hash-Sets. \\[0.2em]
        In particolar, every \textbf{Hash-Structure} has been implemented following two different rules: \textbf{Open hashing} and \textbf{Close hashing}. \vspace{0.5cm}

        \subsection*{\normalsize CONTENTS OF THE LIBRARY} \addcontentsline{toc}{subsection}{CONTENTS OF THE LIBRARY}

            Since the Classes definitions have been made using templates, this is a \textit{header-only} library. \\[0.2em]
            However, in order to be optimized for the most common uses, have also been build some shared-library files \textit{.so}, containing the
            \textbf{explicit template instantations} of some primitive types. \\[0.2em]
            The only precaution to take when compiling the library headers, is to set the flag \textbf{-std=c++20}, since it is needed in order to work properly. \\[0.4cm]
            The library is divided in multiple headers:
            \begin{itemize}

                \item \textbf{"IF\_Hash.hpp":} here is defined the Interface \code{LDN::Hash}, with the derived Abstract Class \code{LDN::HashSet}, which is the Polimorphic 
                    Abstract Class for the \textbf{Hash-Structure}.
                \item \textbf{"Hash.hpp":} the generic header, which includes all the \textbf{Hash-Structure} (it has been designed in order to be included alone).
                \item \textbf{"HashSet.hpp":} the header file whic implements the Specifications for the \code{LDN::HashSet} Abstract Class.
                \item \textbf{"\textit{Specification}.tpp":} template implementation for the Specification Classes (such as \code{LDN::hash\_set::Open}).

            \end{itemize} \vspace{0.5cm}

        \subsection*{\normalsize INHERITARY STRUCTURE} \addcontentsline{toc}{subsection}{INHERITARY STRUCTURE}

            The library starts with the definition of an Interface, represented by a simple abstract class: \textit{"Hash"}. \\[0.2em]
            Starting from this class, all the derivated ones create a Tree Structure, which allows to manage in a simple way all the Abstract \textbf{Hash-Structure}, and makes easy including 
            new ones or derivating more Specifications for the already existing ones. \vspace{0.5cm}

        \subsection*{\normalsize ITERATOR IMPLEMENTATION} \addcontentsline{toc}{subsection}{ITERATOR IMPLEMENTATION}

            In order to be used in a simple and secure way, the \code{LDN::HashSet} Class has been implemented as an \textit{Iterator}, both via Polimorphism and via Specification.

    % Functions Documentation
    \pagebreak
    \section*{Classes Documentation} \addcontentsline{toc}{section}{Classes Documentation}

        In the \textit{"Hash"} library have been implemented a total of four classes:
        \begin{enumerate}
            
            \item[IF.] \textbf{LDN::Hash:} the Root Interface, from which all the other \textbf{Hash-Structure} derive from.
            \item \textbf{LDN::HashSet:} Abstract Class which defines an Hash-Set Data-Structure.
            \begin{enumerate}
                \item \textbf{LDN::hash\_set::Open:} Hash-Set with Open hashing rule.
                \item \textbf{LDN::hash\_set::Close:} Hash-Set with Close hashing rule.
            \end{enumerate}

        \end{enumerate}

        \pagebreak
        \libcategory{IF. Hash}

            \libclass{template <typename TYPE> \\[0.2em]
                class Hash \{ \\[0.4cm]
                \textcolor{black}{.} \quad protected: \\[0.4cm]
                    \textcolor{black}{.} \qquad \quad // attributes \\[0.2em]
                    \textcolor{black}{.} \qquad \quad size\_t size; \\[0.2em]
                    \textcolor{black}{.} \qquad \quad size\_t elements; \\[0.4cm]
                    \textcolor{black}{.} \qquad \quad //Method --- Hash Function \\[0.2em]
                    \textcolor{black}{.} \qquad \quad virtual const size\_t hash(const TYPE\& key) const noexcept = 0; \\[0.4cm]
                \textcolor{black}{.} \quad public: \\[0.4cm]
                    \textcolor{black}{.} \qquad \quad // constructor / destructor \\[0.2em]
                    \textcolor{black}{.} \qquad \quad explicit Hash(const size\_t\& sz) : size(sz), elements(0) \{\} \\[0.2em]
                    \textcolor{black}{.} \qquad \quad virtual \midtilde Hash() noexcept = default; \\[0.4cm]
                    \textcolor{black}{.} \qquad \quad // getters \\[0.2em]
                    \textcolor{black}{.} \qquad \quad inline const size\_t\& getSize() const noexcept \{ return this->size; \} \\[0.2em]
                    \textcolor{black}{.} \qquad \quad inline const size\_t\& getElements() const noexcept \{ return this->elements; \} \\[0.4cm]
                    \textcolor{black}{.} \qquad \quad // Methods --- clear / contains / toString \\[0.2em]
                    \textcolor{black}{.} \qquad \quad virtual void clear() noexcept = 0; \\[0.2em]
                    \textcolor{black}{.} \qquad \quad virtual bool contains(const TYPE\& key) const noexcept = 0; \\[0.2em]
                    \textcolor{black}{.} \qquad \quad virtual std::string toString() const noexcept = 0; \\[0.4cm]
                \}; \\[0.4cm]
                // output stream operator << \\[0.2em]
                template <typename TYPE> \\[0.2em]
                inline std::ostream\& operator<<(std::ostream\& out\_stream, const LDN::Hash<TYPE>\& hash) \{ \\[0.2em]
                    \textcolor{black}{.} \quad out\_stream << hash.toString(); \\[0.2em]
                    \textcolor{black}{.} \quad return out\_stream; \\[0.2em]
                \}
            }
                
                \libfunction{size\_t size;} \noindent
                The \textbf{Hash-Structure} size, it is a \textit{protected} member, since the user is not able to modify it.

                \libfunction{size\_t elements;} \noindent
                The number of elements currently stored in the \textbf{Hash-Structure}, it is a \textit{protected} member, since the user is not able to modify it.

                \pagebreak
                \libfunction{explicit Hash(const size\_t\& sz);} \noindent
                This is the \textit{constructor}, which assignes a size\_t \code{sz} to the \textit{protected} member \code{Hash.size}, then sets to 0 the \textit{protected} member \code{Hash.elements}.

                \libfunction{virtual \midtilde Hash() noexcept = default;} \noindent
                This is the \textit{destructor}, which is automatically invocated when deleting the object. \\[0.2em]
                It has been left as a \textit{virtual method}, in order to ensure the Class to be abstract.

                \libfunction{virtual void clear() noexcept = 0;} \noindent
                This \textit{virtual method} imposes that each \textbf{Hash-Structure} has to provide a method to clean itself.
                
                \libfunction{virtual bool contains(const TYPE\& key) const noexcept = 0;} \noindent
                This \textit{virtual method} imposes that each \textbf{Hash-Structure} has to provide a method to check if an element is stored.
                
                \libfunction{virtual std::string toString() const noexcept = 0;} \noindent
                This \textit{virtual method} imposes that each \textbf{Hash-Structure} has to provide a method to be printed correctly.

                \libfunction{inline std::ostream\& operator<<(std::ostream\& out\_stream, const LDN::Hash<TYPE>\& hash);} \noindent
                Operator related to every \textbf{Hash-Structure}, which allows to print it via \code{std::ostream} using its method \code{Hash.toString()}.

        \pagebreak
        \libcategory{1. HashSet}

            \libclass{template <typename TYPE> \\[0.2em]
                class HashSet : public Hash<TYPE> \{ \\[0.4cm]
                \textcolor{black}{.} \quad public: \\[0.4cm]
                    \textcolor{black}{.} \qquad \quad // constructor / destructor \\[0.2em]
                    \textcolor{black}{.} \qquad \quad explicit HashSet(const size\_t\& sz) : Hash<TYPE>(sz) \{\} \\[0.2em]
                    \textcolor{black}{.} \qquad \quad virtual \midtilde HashSet() noexcept = default; \\[0.4cm]
                    \textcolor{black}{.} \qquad \quad // Methods --- insert / remove / resize /loadFactor \\[0.2em]
                    \textcolor{black}{.} \qquad \quad virtual void insert(const TYPE\& key) = 0; \\[0.2em]
                    \textcolor{black}{.} \qquad \quad virtual void insert(const TYPE\& key) noexcept = 0; \\[0.2em]
                    \textcolor{black}{.} \qquad \quad virtual void resize() noexcept = 0; \\[0.2em]
                    \textcolor{black}{.} \qquad \quad inline const double loadFactor() const \{ \\[0.2em]
                        \textcolor{black}{.} \qquad \qquad \quad return static\_cast<double>(this->elements) / static\_cast<double>(this->size); \\[0.2em]
                    \textcolor{black}{.} \qquad \quad \} \\[0.4cm]
                    \textcolor{black}{.} \qquad \quad // ------------------------- Iterator Interface ------------------------- \\[0.2em]
                    \textcolor{black}{.} \qquad \quad ... \\[0.4cm]
                    \textcolor{black}{.} \qquad \quad // Iterator Methods --- begin / end \\[0.2em]
                    \textcolor{black}{.} \qquad \quad virtual Iterator begin() noexcept = 0; \\[0.2em]
                    \textcolor{black}{.} \qquad \quad virtual Iterator end() noexcept = 0; \\[0.4cm]
                \};
            }

                \libfunction{explicit HashSet(const size\_t\& sz);} \noindent
                This is the \textit{constructor}, which calls back the constructor of the Interface \code{LDN::Hash<TYPE>}.

                \libfunction{virtual \midtilde HashSet() noexcept = default;} \noindent
                This is the \textit{destructor}, which is automatically invocated when deleting the object. \\[0.2em]
                It has been left as a \textit{virtual method}, in order to ensure the Class to be abstract.

                \libfunction{virtual void insert(const TYPE\& key) = 0;} \noindent
                This \textit{virtual method} imposes that each \textbf{Hash-Set} has to provide a method to insert and store correctly a key (if possible).
                
                \libfunction{virtual void remove(const TYPE\& key) noexcept = 0;} \noindent
                This \textit{virtual method} imposes that each \textbf{Hash-Set} has to provide a method to remove a key (if already stored).
                
                \pagebreak
                \libfunction{virtual void resize() noexcept = 0;} \noindent
                This \textit{virtual method} imposes that each \textbf{Hash-Set} has to provide a method to resize itself.

                \libfunction{inline const double loadFactor() const} \noindent
                Method which allows to calculate the current \textit{load factor} of the \textbf{Hash-Set}.

                \libfunction{// ------------------------- Iterator Interface -------------------------} \noindent
                Technical implementation of the \textbf{Hash-Set} as an \textit{Iterator}, if you want to see the technicalities, check the \textit{header "IF\_Hash.hpp"}.

                \libclass{virtual Iterator begin() noexcept = 0; \\[0.2em]
                    virtual Iterator end() noexcept = 0;} \noindent
                Method which allows the \textbf{Hash-Set} to be used as an \textit{Iterator}, the user does not need to call them directly. \\[0.2em]
                They have been left as \textit{virtual methods} since the implementation differs depending on the Specification.

        \pagebreak
        \libcategory{1.1 HashSet --> Open}

            \libclass{template <typename TYPE> \\[0.2em]
                class Open : public LDN::HashSet<TYPE> \{ \\[0.4cm]
                \textcolor{black}{.} \quad private: \\[0.4cm]
                    \textcolor{black}{.} \qquad \quad // helper Bucket definition \\[0.2em]
                    \textcolor{black}{.} \qquad \quad struct Bucket\{ \\[0.4cm]
                        \textcolor{black}{.} \qquad \qquad \quad TYPE value; \\[0.2em]
                        \textcolor{black}{.} \qquad \qquad \quad struct Bucket* next; \\[0.4cm]
                        \textcolor{black}{.} \qquad \qquad \quad explicit Bucket(const TYPE\& key): value(key), next(nullptr) \{\} \\[0.2em]
                        \textcolor{black}{.} \qquad \qquad \quad \midtilde Bucket() noexcept = default; \\[0.4cm]
                    \textcolor{black}{.} \qquad \quad \}; \\[0.4cm]
                    \textcolor{black}{.} \qquad \quad// attributes \\[0.2em]
                    \textcolor{black}{.} \qquad \quad std::unique\_ptr<Bucket*[]> table; \\[0.4cm]
                    \textcolor{black}{.} \qquad \quad // Method --- Hash Function \\[0.2em]
                    \textcolor{black}{.} \qquad \quad const size\_t hash(const TYPE\& key) const noexcept override; \\[0.4cm]
                \textcolor{black}{.} \quad public: \\[0.4cm]
                    \textcolor{black}{.} \qquad \quad // constructor / destructor \\[0.2em]
                    \textcolor{black}{.} \qquad \quad explicit Open(const size\_t\& sz); \\[0.2em]
                    \textcolor{black}{.} \qquad \quad \midtilde Open() noexcept override; \\[0.4cm]
                    \textcolor{black}{.} \qquad \quad // Hash Methods --- implementation \\[0.2em]
                    \textcolor{black}{.} \qquad \quad ... \\[0.4cm]
                    \textcolor{black}{.} \qquad \quad // HashSet Methods --- implementation \\[0.2em]
                    \textcolor{black}{.} \qquad \quad ...\\[0.4cm]
                    \textcolor{black}{.} \qquad \quad // ------------------------- Iterator Implementation ------------------------- \\[0.2em]
                    \textcolor{black}{.} \qquad \quad ... \\[0.4cm]
                    \textcolor{black}{.} \qquad \quad // Iterator Methods --- begin / end \\[0.2em]
                    \textcolor{black}{.} \qquad \quad LDN::HashSet<TYPE>::Iterator begin() noexcept override \{ \\[0.2em]
                        \textcolor{black}{.} \qquad \qquad \quad SpecificationIterator it(this->table.get(), this->size, 0); \\[0.2em]
                        \textcolor{black}{.} \qquad \qquad \quad return typename LDN::HashSet<TYPE>::Iterator(it.clone()); \\[0.2em]
                    \textcolor{black}{.} \qquad \quad \} \\[0.2em]
                    \textcolor{black}{.} \qquad \quad LDN::HashSet<TYPE>::Iterator end() noexcept override \{ \\[0.2em]
                        \textcolor{black}{.} \qquad \qquad \quad SpecificationIterator it(this->table.get(), this->size, this->size); \\[0.2em]
                        \textcolor{black}{.} \qquad \qquad \quad return typename LDN::HashSet<TYPE>::Iterator(it.clone()); \\[0.2em]
                    \textcolor{black}{.} \qquad \quad \} \\[0.4cm]
                \};
            }

                \libfunction{struct Bucket\{...\};} \noindent
                Definition of an helper struct \code{LDN::hash\_set::Open::Bucket}, private to the \textbf{Hash-Set} Specification, which allows to use the Hash-Structure with 
                chaining (following the \textbf{Open hashing} rule).

                \libfunction{std::unique\_ptr<Bucket*[]> table;} \noindent
                Implementation of the Hash-Table used for the \textbf{Hash-Set}, it has been used a \code{std::unique\_ptr<>} in order to keep the user from accessing the table 
                directly (due to the concept of \textbf{Hash-Set}).

                \libfunction{const size\_t hash(const TYPE\& key) const noexcept override;} \noindent
                Implementation of the \textit{hash function}, it is based on the standard-library struct \code{std::hash<>}, allowing correct and complete management of the 
                hashing rule following \textbf{C++} standard.

                \libfunction{explicit Open(const size\_t\& sz);} \noindent
                This is the \textit{constructor}, which calls back the constructor of the Generalization \code{LDN::HashSet<TYPE>} and makes an unique Hash-Table.

                \libfunction{\midtilde Open() noexcept override;} \noindent
                This is the \textit{destructor}, which is automatically invocated when deleting the object.

                \libfunction{// Hash Methods --- implementation} \noindent
                Implementation of the Methods declared by the Interface \code{LDN::Hash<TYPE>}.

                \libfunction{// HashSet Methods --- implementation} \noindent
                Implementation of the Methods declared by the Generalization \code{LDN::HashSet<TYPE>}.
                
                \libfunction{// ------------------------- Iterator Implementation -------------------------} \noindent
                Technical implementation of the \textbf{Hash-Set} as an \textit{Iterator}, if you want to see the technicalities, check the \textit{header "HashSet.hpp"}.

                \libclass{LDN::HashSet<TYPE>::Iterator begin() noexcept override; \\[0.2em]
                    LDN::HashSet<TYPE>::Iterator end() noexcept override;} \noindent
                Method which allows the \textbf{Hash-Set} to be used as an \textit{Iterator}, the user does not need to call them directly. \\[0.2em]
                They implement the Polimorphic \textbf{Hash-Set} ones, so that both Generalization and Specification can be used as \textit{Iterators}.

        \pagebreak
        \libcategory{1.2 HashSet --> Close}

            \libclass{template <typename TYPE> \\[0.2em]
                class Close : public LDN::HashSet<TYPE> \{ \\[0.4cm]
                \textcolor{black}{.} \quad private: \\[0.4cm]
                    \textcolor{black}{.} \qquad \quad // helper fake-constants declaration \\[0.2em]
                    \textcolor{black}{.} \qquad \quad TYPE EMPTY; \\[0.2em]
                    \textcolor{black}{.} \qquad \quad TYPE TOMBSTONE; \\[0.4cm]
                    \textcolor{black}{.} \qquad \quad // attributes \\[0.2em]
                    \textcolor{black}{.} \qquad \quad std::unique\_ptr<TYPE[]> table; \\[0.2em]
                    \textcolor{black}{.} \qquad \quad bool quadratic; \\[0.4cm]
                    \textcolor{black}{.} \qquad \quad // Method --- Hash Function \\[0.2em]
                    \textcolor{black}{.} \qquad \quad const size\_t hash(const TYPE\& key) const noexcept override; \\[0.4cm]
                \textcolor{black}{.} \quad public: \\[0.4cm]
                    \textcolor{black}{.} \qquad \quad // constructor / destructor \\[0.2em]
                    \textcolor{black}{.} \qquad \quad explicit Close(const size\_t\& sz, bool probing, TYPE empty, TYPE tombstone); \\[0.2em]
                    \textcolor{black}{.} \qquad \quad \midtilde Close() noexcept override = default; \\[0.4cm]
                    \textcolor{black}{.} \qquad \quad // Hash Methods --- implementation \\[0.2em]
                    \textcolor{black}{.} \qquad \quad ... \\[0.4cm]
                    \textcolor{black}{.} \qquad \quad // HashSet Methods --- implementation \\[0.2em]
                    \textcolor{black}{.} \qquad \quad ...\\[0.4cm]
                    \textcolor{black}{.} \qquad \quad // ------------------------- Iterator Implementation ------------------------- \\[0.2em]
                    \textcolor{black}{.} \qquad \quad ... \\[0.4cm]
                    \textcolor{black}{.} \qquad \quad // Iterator Methods --- begin / end \\[0.2em]
                    \textcolor{black}{.} \qquad \quad LDN::HashSet<TYPE>::Iterator begin() noexcept override \{...\} \\[0.2em]
                    \textcolor{black}{.} \qquad \quad LDN::HashSet<TYPE>::Iterator end() noexcept override \{...\} \\[0.4cm]
                \};
            }

                \pagebreak
                \libfunction{TYPE EMPTY;} \noindent
                Declaration of an helper fake-constant, private to an istance of the \textbf{Hash-Set} Specification, which allows to use the Hash-Structure as a single array 
                (following the \textbf{Close hashing} rule). \\[0.2em]
                It is set by the \textit{constructor}.

                \libfunction{TYPE TOMBSTONE;} \noindent
                Declaration of an helper fake-constant, private to an istance of the \textbf{Hash-Set} Specification, which allows to use the Hash-Structure as a single array 
                (following the \textbf{Close hashing} rule). \\[0.2em]
                It is set by the \textit{constructor}.

                \libfunction{std::unique\_ptr<TYPE[]> table;} \noindent
                Implementation of the Hash-Table used for the \textbf{Hash-Set}, it has been used a \code{std::unique\_ptr<>} in order to keep the user from accessing the table 
                directly (due to the concept of \textbf{Hash-Set}).

                \libfunction{bool quadratic;} \noindent
                Flag wich represents the method of probing (linear if set to \textit{false}, quadratic is set to \textit{true}).

                \libfunction{const size\_t hash(const TYPE\& key) const noexcept override;} \noindent
                Implementation of the \textit{hash function}, it is based on the standard-library struct \code{std::hash<>}, allowing correct and complete management of the 
                hashing rule following \textbf{C++} standard.

                \libfunction{explicit Close(const size\_t\& sz, bool probing, TYPE empty, TYPE tombstone);} \noindent
                This is the \textit{constructor}, which calls back the constructor of the Generalization \code{LDN::HashSet<TYPE>} and makes an unique Hash-Table. \\[0.2em]
                It also set for the only time the fake-constants \code{EMPTY} and \code{TOMBSTONE}.

                \libfunction{\midtilde Close() noexcept override = default;} \noindent
                This is the \textit{destructor}, which is automatically invocated when deleting the object. \\[0.2em]
                It is set to default, since the \code{std::unique\_ptr<>} delete itself automatically, and there are no other deletions (like chaining-related ones).

                \libfunction{// Hash Methods --- implementation} \noindent
                Implementation of the Methods declared by the Interface \code{LDN::Hash<TYPE>}.

                \libfunction{// HashSet Methods --- implementation} \noindent
                Implementation of the Methods declared by the Generalization \code{LDN::HashSet<TYPE>}.
                
                \pagebreak
                \libfunction{// ------------------------- Iterator Implementation -------------------------} \noindent
                Technical implementation of the \textbf{Hash-Set} as an \textit{Iterator}, if you want to see the technicalities, check the \textit{header "HashSet.hpp"}.

                \libclass{LDN::HashSet<TYPE>::Iterator begin() noexcept override; \\[0.2em]
                    LDN::HashSet<TYPE>::Iterator end() noexcept override;} \noindent
                Method which allows the \textbf{Hash-Set} to be used as an \textit{Iterator}, the user does not need to call them directly. \\[0.2em]
                They implement the Polimorphic \textbf{Hash-Set} ones, so that both Generalization and Specification can be used as \textit{Iterators}.

\end{document}